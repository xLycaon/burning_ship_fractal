% Diese Zeile bitte -nicht- aendern.
\documentclass[course=erap]{aspdoc}

%%%%%%%%%%%%%%%%%%%%%%%%%%%%%%%%%
%% TODO: Ersetzen Sie in den folgenden Zeilen die entsprechenden -Texte-
%% mit den richtigen Werten.
\newcommand{\theGroup}{146} % Beispiel: 42
\newcommand{\theNumber}{A217} % Beispiel: A123
\author{Giancarlo Calvache \and Vorname2 Nachname2 \and Vorname3 Nachname3}
\date{Wintersemester 2022/23} % Beispiel: Wintersemester 2019/20
%%%%%%%%%%%%%%%%%%%%%%%%%%%%%%%%%

% Diese Zeile bitte -nicht- aendern.
\title{Gruppe \theGroup{} -- Abgabe zu Aufgabe \theNumber}

\begin{document}
\maketitle

\section{Einleitung}

Diese Ausarbeitung befasst sich mit der computergestützten Darstellung von komplexen Fraktalen.
Fraktale sind geometrische Figuren, die sich mittels gewöhnlicher euklidischer Geometrie nicht beschreiben lassen können.
Ein Zitat des Vaters der fraktalen Geometrie, Benoit Mandelbrot, verdeutlicht diesen Unterschied: "Clouds are not spheres, mountains are not cones
coastlines are not circles...nor does lightning travel in a straight line."
Zusammengefasst lässt sich ein Fraktal als eine geometrische Figur vorstellen, die ihre Komplexität bewahrt,
egal wie sehr das Fraktal an einer bestimmten Stelle vergrößert wird.

Diese Eigenschaft der Komplexitätbewahrung wird genutzt, um möglichst viele Details des Fraktals vergrößert darzustellen.
Die Problematik liegt hierbei in begrenzten Registergrößen.
Das ist Ziel es, eine optimisierte Implementierung für die Iterationsvorschrift des Burning-Ship Fraktals zu schreiben und ein Rahmenprogramm zu entwickeln,
durch welches das Burning-Ship-Fraktal in hoher Auflösung als verlustlose Bilddatei generiert werden kann. Hierzu verwenden wir das kompressionsfreie Format BMP mit einer Farbtiefe von 8-bit. Darüberhinaus, wird das Rahmenprogramm durch den
Zusatz erweitert, das Fraktal bis zu X-Fach vergrößert erzeugen zu können, um die meist verborgene geometrische Struktur von Fraktalen zu verdeutlichen.

\section{Lösungsansatz}


% TODO: Je nach Aufgabenstellung einen der Begriffe wählen
\section{Korrektheit/Genauigkeit}

Die Genauigkeit der berechneten Bilder wird.

\section{Performanzanalyse}


\section{Zusammenfassung und Ausblick}

% TODO: Fuegen Sie Ihre Quellen der Datei Ausarbeitung.bib hinzu
% Referenzieren Sie diese dann mit \cite{}.
% Beispiel: CR2 ist ein Register der x86-Architektur~\cite{intel2017man}.
\bibliographystyle{plain}
\bibliography{Ausarbeitung}{}

\end{document}
